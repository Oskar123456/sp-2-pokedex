\documentclass[leqno,12pt]{article}
\usepackage{amsmath,amssymb}
\usepackage{graphicx}
\usepackage{hyperref}
\usepackage{floatrow}
\usepackage{minted}    
\floatsetup[listing]{style=Plaintop}    
\usepackage[most]{tcolorbox}
\usepackage{float}
\restylefloat{table}

\usepackage{xcolor}
\hypersetup{
    colorlinks,
    linkcolor={red!50!black},
    citecolor={blue!50!black},
    urlcolor={blue!80!black}
}

\newcommand{\Term}{Efterår 2024}
\newcommand{\Course}{Datamatiker}

\newcommand{\Assignment}{sp-2 api-dokumentation}
\newcommand{\DueDate}{tirsdag d. 22 oktober}

\usepackage[body={6in,9in}]{geometry}

\newcommand{\brak}[1]{\left\langle #1\right\rangle}
\newcommand{\abs}[1]{\left\lvert #1\right\rvert}

\DeclareMathOperator{\lcm}{lcm}

\newcommand{\Neg}{\phantom{-}}
\renewcommand{\emptyset}{\varnothing}
\renewcommand{\subset}{\subseteq}

\newcommand{\ds}{\displaystyle}

%% Letter symbols
\newcommand{\Number}[1]{\mathbf{#1}}
\newcommand{\C}{\Number{C}}
\newcommand{\N}{\Number{N}}
\newcommand{\Q}{\Number{Q}}
\newcommand{\R}{\Number{R}}
\newcommand{\Z}{\Number{Z}}

\newcommand{\eps}{\varepsilon}

\newcommand{\Input}[1]{\includegraphics{#1.pdf}}

\newcommand{\Prob}[1]{\noindent\textbf{Question #1:}\quad}
\newcommand{\soln}[1][Answer]{\noindent\textbf{#1}\quad}

\newcommand{\Cmd}[1]{\textbackslash\texttt{#1}}

\pagestyle{empty}

\usepackage{xcolor,colortbl}

\newcommand{\mc}[2]{\multicolumn{#1}{c}{#2}}
\definecolor{Gray}{gray}{0.85}
\definecolor{LightGray}{rgb}{0.88,0.88,0.88}

\begin{document} 

Oskar Bahner Hansen (\texttt{cph-oh82@cphbusiness.dk})
\begin{center}

\textbf{Copenhagen Business School \Term} \\
\textbf{\Course} 3. semester \\
\textbf{\Assignment} søndag d. 27. oktober
\end{center}

\subsection*{\underline{Beskrivelse}}
\textbf{Titel}: \normalsize \texttt{pokedex}. \textbf{url}: \texttt{\href{http://pokedex.obhnothing.dk}{pokedex.obhnothing.dk}}. \\ \textbf{Github}: \normalsize \texttt{https://github.com/Oskar123456/sp-2-pokedex} \newline \newline
    \textit{En API til at hente information om pokemons, som f.eks. kunne bruges til et spil. Med en backend der henter disse informationer fra}
    \texttt{https://pokeapi.co}.
\subsection*{\underline{API}}

%%\begin{table}[h]
%%\hspace*{1in}
\begin{tabular}{l | l | l | l | l | l}
\rowcolor{white}
    \multicolumn{1}{c}{\texttt{method}} & \multicolumn{1}{c}{url} & \multicolumn{1}{c}{params} & \multicolumn{1}{c}{request body} & \multicolumn{1}{c}{response} & \multicolumn{1}{c}{error} \\
\hline
\rowcolor{Gray}
        \texttt{GET}    & \texttt{/api/pokemon}& \texttt{id}  &  & \texttt{pokejson} & \texttt{errorjson} \\
\hline
        \texttt{GET}    & \texttt{/api/pokemon/all}& &  & \texttt{pokejson[]} & \texttt{errorjson} \\
\hline
\rowcolor{Gray}
        \texttt{POST}   & \texttt{/api/pokemon} & & \texttt{pokejson}~(-- id) & \texttt{pokejson} & \texttt{errorjson} \\
\hline
        \texttt{PUT} & \texttt{/api/pokemon} & & \texttt{pokejson} & \texttt{pokejson} &  \texttt{errorjson} \\
\hline
\rowcolor{Gray}
        \texttt{DELETE} & \texttt{/api/pokemon} & \texttt{id} & & \texttt{pokejson} & \texttt{errorjson} \\
\hline
\end{tabular}
%%\end{table}

\subsubsection*{Formater}
 -- \texttt{pokejson}-format:
\begin{listing}[ht!]
\begin{minted}[fontsize=\footnotesize]{js}
    { "id" : Number,
      "name" : String,
      "basexp" : Number,
      "height" : Number,
      "isdefault" : Boolean,
      "order" : Number,
      "weight" : Number,
      "abilities" : Object[], 
      "forms" : Object[],
      "locations" : Object[],
      "moves" : Object[],
      "sprites" : String[],
      "species" : Object,
      "stats" : Object[],
      "types" : String[]}
\end{minted}
\end{listing}

 -- \texttt{errorjson}-format:
\begin{listing}[ht!]
\begin{minted}[fontsize=\footnotesize]{js}
    {  "status" : Number, "message" : String }
\end{minted}
\end{listing}


\end{document}


























